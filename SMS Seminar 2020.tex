\documentclass{beamer}
\usepackage{beamerthemesplit, textpos, wrapfig}
\usetheme{Warsaw}

\definecolor{uncblue}{RGB}{1, 45, 91}
\definecolor{uncgold}{RGB}{254,189,54}
\setbeamercolor{structure}{fg=uncblue}
\setbeamercolor{frametitle}{fg=uncgold}

\bibliographystyle{plain}
\usepackage{amsmath, amssymb, amscd}
%\usepackage[metapost]{mfpic}

\newcommand{\bra}{\langle}
\newcommand{\ket}{\rangle}
\newtheorem*{thm}{Theorem}
\newtheorem*{lem}{Lemma}
\newtheorem*{prop}{Proposition}
\newtheorem*{question}{Question}

\setbeamertemplate{blocks}[rounded][shadow=false]
\usenavigationsymbolstemplate{}
\setbeamertemplate{footline}[frame number]

\title{\textcolor{uncgold}{Standards Based Grading} }
\author{Virgil Pierce (University of Northern Colorado)}
\date{August 7, 2018}

\begin{document}

\begin{frame}
\titlepage

\end{frame}

\begin{frame}{What is Standards Based Grading (SBG)}

Basic features of a SBG class:
\begin{itemize}

\item Students are provided with a list of standards on the course syllabus. Standards are measurable statements of things students will do. 

\item All class activities are organized around the standards in a clearly identified way.  Homework, videos, examples, etc. Everything has an identified standard attached. (think Quality Matters)

\item Students are given an opportunity to prepare to demonstrate their mastery on a standard.

\item Students are given multiple opportunities to demonstrate their mastery on a standard, often involving a choice of which standard they want to demonstrate.

\item In most cases, once they have demonstrated their mastery, they are done with that standard.

\end{itemize}

\end{frame}

\begin{frame}{Who is using SBG}

SBG is most frequently used in high schools, and it is one of the techniques recommended by the College Board for AP Calculus and AP Statistics.

\medskip

You will find a number of Calculus 1-3 courses around the country using it, and most of the design of my syllabus was lifted from their materials. 

\medskip

I follow a number of instructors on Twitter that discuss SBG, and it was a blog by Kate Owens at the College of Charleston about SBG in Linear Algebra that got me started.  \url{https://blogs.cofc.edu/owensks/2015/04/01/sbg-docs/}
\end{frame}


\begin{frame}{SBG Advanced Linear Algebra}

\only<1>{
\begin{block}{The Standards}
21 Standards were organized around 6 {\it big questions}.  Example:
\begin{enumerate}

\item[3.] How do we classify linear transformations by their structure?
\begin{enumerate}

\item[a.] I can find invariant subspaces, minimal polynomials, and characteristic polynomials of linear transformations over a variety of fields.

\item[b.] I can find a matrix with a given characteristic and minimal polynomial. I can describe all matrices with these polynomials.

\item[c.] I can find the rational cannonical form of a linear transformation.

\item[d.] I can find the Jordan cannonical form of a linear transformation.

\end{enumerate}

\end{enumerate}

\end{block}
}

\only<2>{
\begin{block}{The Standards}
21 Standards were organized around 6 {\it big questions}.  Example:
\begin{enumerate}

\item[6.] What is the determinant of a linear transformation?
\begin{enumerate}

\item[a.] I can explain the construction of exterior algebras and give a description of the resulting vector spaces.

\item[b.] I can explain the definition of the determinant in terms of exterior algebras and why it agrees with the one we found in previous classes.

\item[c.] I can show from our definition that a determinant is multilinear, 0 for singular linear transformations, corresponds to area or volume in 2 and 3 dimensions, and that the determinant of the inverse of a matrix is the reciprocal of the determinant.

\end{enumerate}

\end{enumerate}

\end{block}
}



\only<3>{
\begin{block}{Demonstrating Mastery}

\begin{itemize}

\item Students had a daily quiz - quizzes were offered on most of the standards that we had covered to date (with new versions produced frequently) and they then choose which one to attempt.

\item The course also had two midterms and a final exam which had questions from most standards available.  Students could attempt any standard they had not yet demonstrated or exceeded mastery of - or not, it is up to them.

\item Once students have demonstrated mastery they are done with that standard.

\end{itemize}

\end{block}
}

\only<4>{
\begin{block}{Assessments}

Standards were a mix of computational, conceptual, and proof.  

\medskip

For computational and conceptual standards new versions of the quizzes and questions on the exams were generated frequently, for proofs less often.

\medskip

Some standards required multi question assessments (parts).

\medskip

{\bf Attempts at standards are assessed on a simple rubric:  
\begin{itemize} 
\item Incomplete or Inappropriate Submission, 
\item Needs Revision, 
\item Meets Standard, 
\item Exceeds Standard. \end{itemize}} 

\end{block}
}

\only<5>{
\begin{block}{Final Grades}
The final grade is based upon how close the student came to mastering or exceeding all standards. 
\end{block}
}


\end{frame}

\begin{frame}{Why Standards Based Grading}

\begin{itemize}

\only<1>{
\item {\bf Changing the Conversation with Students:}
A major frustration for me in teaching, particularly advanced courses, has been assigning / negotiating {\it points} with students. 

\item {\bf Providing Feedback versus Grading:}
I've realized that {\it points} has not been giving students feedback that helps them grow.

\item {\bf Helping Every Student Grow:}
By focusing on what students need to accomplish, I've found it easier to work with each individual student on what they need to be doing to continue their growth as a mathematician.}

\only<2>{
\item {\bf Faster Grading:}
The grading has been faster and more efficient.  I get a lot fewer nonsense submissions than I have in past proof-based courses; and usually after one {\it Needs Revision} a student can complete a standard.

\item {\bf Giving Students Agency:}
Studies suggest that giving students agency (or autonomy) over their learning improves learning. Some data indicates the effect is more pronounced with minority and first generation students - those that are less likely to already have developed agency over their learning.}


\only<3>{
\item {\bf Grades:}
I value very much that students' grades are not determined by one test, and that they have many opportunities to demonstrate their mastery and change their situation. Yet at the same time, I'm finding my students are much more aware of how they are doing in the class - they know what standards they are struggling with, and they have identified activities to work on to improve.

\medskip

It was noteworthy to my class that they get credit for what they achieve, not penalized for their mistakes.}

\end{itemize}

\end{frame}

\begin{frame}{Downsides SBG}
\subtitle{and Lessons from my first attempt with Adv. Lin. Alg.}
\begin{itemize}

\only<1>{
\item {\bf So many Quizzes:} The class has 21 standards (on the edge of too many), and each standard will end up having multiple quizzes as students make attempts at them at different times. The median number of quizzes per standard is 3, but for some I've written as many as 5 versions.

\medskip

For the most part this is a one-time cost for the first or second time you teach a particular course with SBG.  Once you have a library of quizzes and questions for the standards it should go better.  For Calculus it should also take much less time to write quizzes.
}

\only<2>{
\item {\bf De-Incentivize Homework:} If you do SBG straight up, it will effectively de-value homework at least as a direct portion of a students grade. For the Adv. Lin. Alg. course this was not a big concern of mine.  For a lower level class, including homework in the final grade, or flipping the class will be necessary to counter this (in my opinion).

\item {\bf Student Resistance:}   (Some) students are very comfortable with traditional course designs. As with flipped and emporium classes be ready for some student resistance. 
}

\end{itemize}

\end{frame}

\begin{frame}{SBG for Calculus}

I had some specific concerns as I've been thinking about how this would work in a lower level class. I thought I would share with you some of what I saw in looking at how people are using SBG in Calculus courses:
\begin{itemize}

\only<1>{
\item {\bf Collaborative Learning:} as mentioned above, SBG will effectively value homework only indirectly. So including extensive in class collaborative learning activities that get students working during class is crucial. 

\item {\bf Quizzes:} Most have students identify ahead of time which standard they intend to demonstrate on a weekly or bi-weekly quiz, some will restrict which standards are available for quizzes.  Grading daily quizzes with 60 students is probably too much - so weekly quizzes would be better.

\item {\bf So Many Quizzes:} One idea is to have students write a quiz for a standard in order to achieve the ``Exceeds Mastery" category.
}

\only<2>{
\item {\bf Pre-Requisites:} Lower level classes are usually pre-requisites for further classes. I have some concerns about checking that students can demonstrate their mastery throughout the class. Some ways people deal with this:
\begin{itemize}

\item Exceed Standard is only available on midterms or the final.

\item Standards + Final determines grade.

\item Standards need to be demonstrated at least once on a midterm or final before students are allowed to call it done.

\item A set of Core Standards that students must over perform on.

\item Checkout Activity (like the AP test for AP Calculus).

\item {\bf Build standards later in the class that will incorporate standards from earlier.}

\end{itemize}
}

\end{itemize}

\end{frame}

\begin{frame}{Student Responses}

Still waiting on the student evaluations.  However here is some of what they have told me:

\begin{enumerate}

\item Physics Major:  ``I like that if I see something interesting on a quiz or midterm I can take the time to explore it without risking my grade in the class."

\item Mathematics Major: ``I've appreciated and needed the multiple attempts to achieve a standard, and that the wrong answers and mistakes do not penalize me."

\end{enumerate}

\end{frame}


\end{document}